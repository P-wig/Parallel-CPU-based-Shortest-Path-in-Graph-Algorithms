\documentclass{article}

\usepackage{arxiv}
\usepackage[utf8]{inputenc}
\usepackage[T1]{fontenc}
\usepackage{hyperref}
\usepackage{url}
\usepackage{booktabs}
\usepackage{amsfonts}
\usepackage{nicefrac}
\usepackage{microtype}
\usepackage{graphicx}
\usepackage{natbib}
\usepackage{doi}
\usepackage{fancyhdr}

\title{Parallel-CPU-based-Shortest-Path-in-Graph-Algorithms}
\date{December 1, 2025}

% Uncomment to override  the `A preprint' in the header
%\renewcommand{\headeright}{Technical Report}
%\renewcommand{\undertitle}{Technical Report}
\renewcommand{\shorttitle}{<Short Title>}

\hypersetup{
pdftitle={Parallel-CPU-based-Shortest-Path-in-Graph-Algorithms},
pdfsubject={},
pdfauthor={<Your Name(s)>},
pdfkeywords={},
}

\author{
    Isaac Shepherd \\
	University of Texas at Austin \\
    is23423@my.utexas.edu \\
    \And
    <Author 2 Name> \\
    <Affiliation 2> \\
    <Email 2> \\
    % Add more authors as needed
}

\pagestyle{fancy}
\fancyfoot[C]{\thepage}

\begin{document}
\maketitle

\begin{abstract}
A comparative study of single-source shortest path algorithms (Dijkstra, Bellman-Ford, Delta-Stepping) implemented using MPI, pthreads, and OpenMP. 
Experiments on large DIMACS graph demonstrate the strengths and weaknesses of each approach in terms of scalability and performance. 
Results show that Bellman-Ford scales well on distributed systems, Delta-Stepping provides a brief increase in performance but degrades at higher thread counts, and Dijkstra's algorithm exhibits poor scalability due to its inherently sequential global minimum selection process.
Results provide practical guidance for selecting parallel SSSP algorithms on parallelism models (MPI, OpenMP, pthreads) for graph processing.
\end{abstract}

\keywords{Single-source shortest path \and Dijkstra \and Bellman-Ford \and Delta-Stepping \and MPI \and OpenMP \and pthreads \and Parallel algorithms}

\section{Introduction}
Single-source shortest path (SSSP) algorithms are fundamental in graph theory with widespread applications in fields such as networking, transportation, and scientific engineering. 
As graphs representing real-world applications grow in size and complexity, efficient parallelization of SSSP algorithms becomes increasingly necessary to achieve practical performance. 
In this project, we compare three classic SSSP algorithms—Dijkstra, Bellman-Ford, and Delta-Stepping—implemented using three parallel programming models: MPI, pthreads, and OpenMP. 
The goal is to evaluate the scalability and efficiency of each approach on the `internet.egr` DIMACS benchmark graph, which consists of 124,651 nodes and 387,240 edges and represents a snapshot of the world internet topology at the autonomous systems level. 
These tests provide insights into the strengths and limitations of different parallelization strategies for high-performance graph processing.

\section{Background}
% Background and related work.

\section{Methods}
% Methods, algorithms, and implementation details.

\section{Results}
% Experimental results, tables, and figures.

\section{Discussion}
% Analysis and discussion of results.

\section{Conclusion}
% Conclusion and future work.

\bibliographystyle{unsrtnat}
\bibliography{references}

\end{document}